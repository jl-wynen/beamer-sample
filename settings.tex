% basic text processing/formatting
\usepackage{xltxtra}            % XeTeX packages
\usepackage[english]{babel}

% maths stuff
\usepackage{amsmath}
\usepackage{amsfonts}
\usepackage{amssymb}
\usepackage{mathtools}
\usepackage{slashed}            % (Feynman-) slashed symbols
\usepackage{commath}            % for nicer differentials
\usepackage{dsfont}             % double stroked characters

\usepackage{tikz}               % drawing all manners of things
\usepackage{forest}             % draw trees using TikZ
\usepackage{datetime}           % for custom date and time format
\usepackage{array}              % for matrices in math environment
\usepackage{float}              % place graphics with "H"
\usepackage{braket}             % Dirac notation
\usepackage{placeins}           % FloatBarrier
\usepackage{lmodern}            % to scale font to whatever I want
\usepackage{verbatim}           % print input as it is
\usepackage{xfrac}              % sideways fractions
\usepackage{ccicons}            % creative commons icons

% For textblock, remove showboxes for release.
% Gets loaded by aiphi, but this way we can specify showboxes.
\usepackage[absolute,overlay]{textpos}

% load customisations
\usepackage{aiphi}

%% devel only
% overlay a grid over the slides
% \usepackage[texcoord,grid,gridcolor=black!10,subgridcolor=black!5,gridunit=mm]{eso-pic}
\usepackage{lipsum}


% ---------------------------------------------------------------------------------------
% setup beamer
% ---------------------------------------------------------------------------------------
% load base theme
\usetheme[titleinfoot, toc, rainbow]{aiphi}
% select dark colour theme
\usecolortheme{aiphidark}


% ---------------------------------------------------------------------------------------
% document info
% ---------------------------------------------------------------------------------------
% these settings must be placed after loading the aiphi beamer theme

% The license command doesn't like the \ccby in a \href for some reason.
% But using an extra command to insert it is fine ¯\_(ツ)_/¯
\newcommand{\thelicense}{\href{https://creativecommons.org/licenses/by/4.0}{\ccby}}

\title{Fancy shmancy talk}

% show author and contributors on title page and author + license in footline
\newcommand{\theauthor}{Jan-Lukas Wynen}
\author[\insertlicense\quad\theauthor]{
  \textbf{\theauthor}
  \and M. Sue
  \and M. Mustermann}

\date{\isodate\today}
\institute{Institute for applied \LaTeX{}
  \and Artsy Science School}

% set the logo colour here because \insertlogo overwrites it somehow
\logo{\color{aiphitext}\includegraphics[width=10mm]{logo.pdf}}

% these are optional
\license{\thelicense}
\email{j-l.wynen@hotmail.de}
\ghhandle{jl-wynen}
% \titlegraphic[height=\paperheight]{beta-morph.png}


% ---------------------------------------------------------------------------------------
% TikZ
% ---------------------------------------------------------------------------------------
\usetikzlibrary{tikzmark, calc, arrows, spy}


% ---------------------------------------------------------------------------------------
% abbreviations
% ---------------------------------------------------------------------------------------
\newcommand{\unit}[1]{\,\text{#1}}
\newcommand{\ev}{\,\text{eV}}
\newcommand{\kev}{\,\text{keV}}
\newcommand{\mev}{\,\text{MeV}}
\newcommand{\gev}{\,\text{GeV}}

\DeclareMathOperator{\Tr}{Tr}

%%% Local Variables:
%%% coding: utf-8
%%% mode: latex
%%% TeX-engine: xetex
%%% TeX-master: "main"
%%% End:
