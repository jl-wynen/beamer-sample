\usepackage{ucs}                % more unicode
\usepackage[utf8x]{inputenc}
\usepackage[T1]{fontenc}
\usepackage[english]{babel}

\usepackage{amsmath}
\usepackage{amsfonts}
\usepackage{amssymb}
\usepackage{mathtools}
\usepackage{slashed}
\usepackage{commath}            % for nicer differentials
\usepackage{bm}                 % bold math
\usepackage{dsfont}
\usepackage{latexsym}

\usepackage{tikz}
\usepackage{calc}
\usepackage{datetime}
\usepackage{array}              % for matrices in math environment
\usepackage{float}              % place graphics with "H"
\usepackage{braket}
\usepackage{placeins}           % FloatBarrier
\usepackage{listings}
\usepackage{color}
\usepackage[xparse, skins]{tcolorbox}   % nice coloured boxes
% for textblock, remove showboxes for release
\usepackage[absolute,overlay]{textpos}

%% devel only
% overlay a grid over the slides
% \usepackage[texcoord,grid,gridcolor=black!10,subgridcolor=black!5,gridunit=mm]{eso-pic}
\usepackage{lipsum}


%----------------------------------------------------------
% general info
% ISO date format
\newdateformat{isodate}{\THEYEAR-\twodigit{\THEMONTH}-\twodigit{\THEDAY}}

\newcommand{\theauthor}{Jan-Lukas Wynen}
\newcommand{\theinstitute}{Institute for applied \LaTeX}
\newcommand{\thetitle}{Fancy shmancy talk}
\newcommand{\thesubtitle}{Does this subtitle appear anywhere}
\newcommand{\thedate}{\isodate\today}

\author{\theauthor}
\title{\thetitle}
\date{\thedate}
\institute{\theinstitute}


%----------------------------------------------------------
% setup beamer
\usetheme[titleinfoot, toc]{aiphi}

% \setbeamertemplate{page number in head/foot}[appendixframenumber]


%----------------------------------------------------------
% tcolorbox
% box for itemizes
\NewTColorBox{itembox}{m}{
    title=#1,
    noparskip,  % not using library parskip
    left=1mm, right=3mm, bottom=3mm, top=3mm,  % less for left to account for itemize
    sharp corners,
    boxrule=0.5mm,
    colframe=aiphiblue,
    colback=white
}

% box with a fancy title
\NewTColorBox{fancybox}{m}{
  enhanced,
  title=#1,
  attach boxed title to top center={
    yshift=-0.25mm-\tcboxedtitleheight/2,  % extra shift is 1/2 boxrule
    yshifttext=1ex-\tcboxedtitleheight/2
  },
  boxed title style={ % draw wedges around the title
    boxrule=0.5mm,
    frame code={ \path[tcb fill frame] ([xshift=-4mm]frame.west)
      -- (frame.north west) -- (frame.north east) -- ([xshift=4mm]frame.east)
      -- (frame.south east) -- (frame.south west) -- cycle; },
    interior code={ \path[tcb fill interior] ([xshift=-2.5mm]interior.west)
      -- (interior.north west) -- (interior.north east)
      -- ([xshift=2.5mm]interior.east) -- (interior.south east) -- (interior.south west)
      -- cycle;}
  },
  noparskip,
  left=3mm, right=3mm, bottom=3mm,
  sharp corners,
  boxrule=0.5mm,
  colframe=aiphiviolet,
  colback=white,
  colbacktitle=aiphiviolet!70!white,
  coltitle=white
}

% box showing a tag besides the title
% arguments: title, tag, tag background, tag foreground
\NewTColorBox{tagbox}{m m O{black!80!white} O{white}}{
  title=#1,
  after title={\hfill\colorbox{#3}{\color{#4}#2}},
  boxsep=0.5mm,
  noparskip,
  left=3mm, right=3mm, bottom=3mm, top=3mm,
  sharp corners,
  boxrule=0.5mm,
  colframe=black!80!white,
  colback=white,
  colbacktitle=white,
  coltitle=black
}


% ----------------------------------------------------------
% TikZ
\usetikzlibrary{tikzmark, calc}
% remember all external nodes to reference them later
\tikzstyle{every picture}+=[remember picture]
% use display style math in TikZ nodes
% \everymath{\displaystyle}


%----------------------------------------------------------
% abbreviations
\newcommand{\unit}[1]{\,\text{#1}}
\newcommand{\ev}{\,\text{eV}}
\newcommand{\kev}{\,\text{keV}}
\newcommand{\mev}{\,\text{MeV}}
\newcommand{\gev}{\,\text{GeV}}

\renewcommand{\i}{\mathrm{i}}
\renewcommand{\epsilon}{\varepsilon}

\newcommand{\Tr}{\text{Tr}}


% set typewriter font to courier
\renewcommand{\ttdefault}{pcr}

\lstdefinelanguage{aiphicpp}{
  morekeywords={class, struct, template, typename, decltype, sizeof, static, if, else, for, while, do, goto, static_cast, dynamic_cast, reinterpret_cast, const_cast, constexpr, const, inline and, or, xor, bitand, bitor, and_eq, or_eq, xor_eq, return, throw, try, catch, noexcept, namespace, using, typedef, enum, union, beak, continue, switch, case, break, continue, default, delete, new, operator, public, private, protected, explicit, extern, mutable, volatile, override, virtual, final},
  morekeywords=[2]{void, unsigned, int, long, float, double, bool, char, auto},
  morecomment=[l]{//},
  morecomment=[s]{/*}{*/},
  morestring=[b]",
  morestring=[b]'
}

\lstdefinestyle{aiphilst}{
  language=aiphicpp,  % the default language
  basicstyle=\scriptsize\ttfamily\color[RGB]{211,211,211},  % default font size
  breakatwhitespace=false,  % if true, lines are only broken at whitespaces
  breaklines=true,
  captionpos=b, % place captions under listing
  frame=l,                         % frame around code, show only to the left
  framerule=1.5pt,                 % frame width
  rulecolor=\color{aiphibordeaux!75!white}, % colour of frame
  numbers=left,  % line numbers, one of none, left, right
  numbersep=6pt, % distance of line numbers from code
  numberstyle=\tiny\color{black},  % text style of line numbers
  backgroundcolor=\color[HTML]{282b2e},
  commentstyle=\itshape\color[HTML]{8cbbad},
  keywordstyle={[1]\bfseries\color{aiphiblue!75!white}},
  keywordstyle={[2]\color{aiphigreen!75!white}},
  stringstyle=\color{aiphired!70!white},
  showstringspaces=false,  % show marker at spaces in strings
}

\lstset{style=aiphilst}


\newcommand{\backupbegin}{
   \newcounter{framenumberappendix}
   \setcounter{framenumberappendix}{\value{framenumber}}
}
\newcommand{\backupend}{
   \addtocounter{framenumberappendix}{-\value{framenumber}}
   \addtocounter{framenumber}{\value{framenumberappendix}}
}


% ----------------------------------------------------------
% miscellanious

% remember current item command so we can switch back to it
\let\olditem\item

%%% Local Variables:
%%% mode: latex
%%% TeX-master: "main"
%%% End:
